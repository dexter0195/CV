\sectionTitle{Esperienze Formative}{\faMortarBoard}
%\renewcommand{\labelitemi}{$\bullet$}
\begin{experiences}
  \experience
    {Gennaio 2016}   {Sviluppatore C}{Progetto di Sistemi Operativi}{Universit\`a di Bologna}
    {Settembre 2016} {
                      Realizzazione del Kernel per il sistema operativo Kaya, un progetto Open Source per architettura ARMv7
                      \begin{itemize}
                        \item Programmazione di Sistema
                        \item Implementazione dello Scheduler
                        \item Implementazione dell' I/O
                      \end{itemize}
                    }
                    {C, Programmazione di Sistema}
  \emptySeparator
  \experience
    {Gennaio 2016}   {Sviluppatore Back-end e Front-end}{Progetto di Tecnologie Web}{Universit\`a di Bologna}
    {Settembre 2016} {
                      Realizzazione della piattaforma EasyRash per la revisione dei paper di una conferenza scientifica.
                      \begin{itemize}
                        \item Realizzazione del back-end in Python con il Framework Flask
                        \item Implementazione comunicazione con MongoDB
                        \item Meccanismi di autenticazione sicura token based
                        \item Implementazione delle API REST
                        \item Implementazione del Front-end in Angular Material
                      \end{itemize}
                    }
                    {Python, MongoDB, Flask, REST, Angular}
  \emptySeparator
  \experience
    {Settembre 2016}   {Amministratore di Sistema}{AdmStaff}{Universit\`a di Bologna}
    {Marzo 2018} {
                      Amministrazione di una piccola rete di server che offrono servizi sperimentali agli
                      studenti di Informatica dell'Universit\`a di Bologna.
                      Installazione e manutenzione di:
                      \begin{itemize}
                        \item Mail Server Postfix
                        \item BIND DNS Server
                        \item GitLab CE
                        \item LDAP
                        \item NFS
                        \item 802.1x
                        \item Radius
                      \end{itemize}
                    }
                    {Linux, Networking, Cisco, HP, Administration}
  \emptySeparator
  \experience
    {Settembre 2017}   {\it{Studio di un Intrusion Detection System basato su Packet Sampling}}{Tesi Triennale}{Universit\`a di Bologna}
    {Marzo 2018} {
                      Tesi sull'analisi di rendimento del monitoraggio di una rete utilizzando la tecnologia di packet sampling sFlow e l'IDS
                      Suricata.
                      \begin{itemize}
                        \item Analisi dei requisiti e delle criticit\`a
                        \item Progettazione della proposta di soluzione
                        \item Implementazione di una stuite di testing e simulazione in C, Bash e Python
                        \item Testing estensivo dei casi d'uso
                        \item Popolamento del database Elasticsearch e fine tuning
                        \item Analisi dei risultati ottenuti con Kibana e valutazioni statistiche
                      \end{itemize}
                    }
                    {Linux, Networks, sFlow, NetFlow, Elasticsearch, Kibana, C, Python, Bash}
  \emptySeparator
\end{experiences}
